\PassOptionsToPackage{quiet}{fontspec}  %depress the problem of font
\documentclass{ctexart}
\setmainfont{Times New Roman}
\begin{document}

\begin{itemize}
    \item 三种模式
    \\ 命令行模式 command line (在这个模式下,不能输入文本,可以在文本间移动,删除一行文本,复制粘贴,跳转指定行,撤销操作等),输入模式 insert mode, 底线命令模式 last line mode。
    \item 输入模式下常用命令
    \\ HOME / END \qquad 行首 / 行尾
    \\ Insert \qquad 插入
    \\ Page Up / Page Down \qquad 上 / 下翻页
    \\ gg  \qquad 跳转到文件开头
    \\ G  \qquad 跳转到文件末尾
    \\ dd  \qquad 删除光标所在一整行数据,且临时存储到缓存区 (类似于 剪切)
    \\ p  \qquad 将指定缓存区内容放在光标位置之下;整行文本放在行下面,若是非整行文本则放在光标后面
    \\ P  \qquad 将指定缓冲区的内容放到当前光标的位置之上;整行文本放在行下面,若是非整行文本则放在光标后面
    \\ u  \qquad 撤销最近的修改,针对所有修改
    \\ U  \qquad 撤销对当前行上做的所有修改
    \\ r  \qquad	替换光标位置上的一个字符,但不会进入 insert 模式,先按 r 再按要替换成的字符,也可以 2r 把光标后面两个都替换掉
    \\ .   \qquad 重复上一次的修改 -- 注意该命令是英文符号点
    \\ >>  \qquad 向右移动本行一段距离
    \\ <<  \qquad 向左移动本行一段距离
    \\ yy  \qquad 复制整行文本到临时缓存区 
    \\ Ctrl + f  \qquad 将光标向下翻满屏,通常每次翻屏24行
    \\ Ctrl + b  \qquad 将光标向上翻满屏,通常每次翻屏24行
    \\ Ctrl + d   \qquad  将光标向下翻半屏,通常每次翻屏12行
    \\ Ctrl + u   \qquad  将光标向上翻半屏,通常每次翻屏12行

    \item 输入模式
    \\ i \qquad 在光标的左侧输入文本,光标右侧文本向右移动
    \\ I \qquad 在光标所在行的行首输入文本,相当于在行首执行了 i 命令
    \\ A \qquad 在光标所在行的尾部进行输入文本,相当于在光标行尾执行a命令
    \item 底线命令模式
    \\ :\qquad 从编辑模式进入底线命令模式,\emph{一定要在英文模式下输入:}
    \\ :q \qquad 退出
    \\ :wq \qquad 保存退出
    \\ :\%s/x/y/g  \qquad 将“x”全局替换为“y”
    \\ /word \qquad 查找word
    \\ :w 	\qquad 只保存文件,但不退出 vim 编辑器
    \\ ZZ  \qquad 保存文件且退出 vim 编辑器
    \\ :q! \qquad 不保存文件,直接退出 vim 编辑器
    \\ :start,endd  \qquad 从第 start 行开始到 end 行进行删除操作 最后一个 d 表示删除,如 :3,4d
    \\ :! command  \qquad 暂时离开vim编辑器去到指令模式下执行 command 命令显示结果,如 :!ls
    \\ :/string  \qquad 从文件首部开始查找到尾部;按 n 键可以跳到下一个,N 上一个,另外按 / 键后,按上下键可以找到以前查找的记录,同样的也有记录
    \\ :?string  	\qquad 同上,从当前光标向文件首部查找
    \\:set nu   	\qquad 显示行号 也可以输入 set number
    \\:set nonu    \qquad 取消行号
    \\:set ai/noai   \qquad 开启/关闭自动缩进

    \item vim 中批量注释
    \\批量注释:\begin{itemize}
        \item   ctrl + v 进入块选择模式,然后移动光标选中你要注释的行,
        \item  按大写的 I 进入行首插入模式输入注释符号;例如:\#号或者\/\/(双斜线),
        \item  输入完毕之后,按两下 ESC,Vim 会自动将你选中的所有行首都加上注释。
    \end{itemize}


\item 批量取消注释:
\begin{itemize}
    \item  ctrl + v 进入块选择模式,选中你要删除的行首的注释符号,
    \item 按 d 即可删除注释。
\end{itemize}
\end{itemize}
\end{document}